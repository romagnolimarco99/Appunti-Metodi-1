\chapter{Norma e Topologia}
Prendiamo uno spazio vettoriale $V$ su $\CC$ di dimensione arbitraria, non
necessariamente finita. La discussione varrà anche per spazi definiti su
$\RR$, basta trascurare *(il coniugio) quando compare. Ora equipaggiamo $V$
con una norma.
\begin{definition}[(norma e spazi normati)]
Una \emph{norma} $\| \; \|: V \mapsto \RR$ è una funzione che ad ogni elemento
$v \in V$ associa un numero reale $\|v\|$, da intendersi come una nozione di
"lunghezza" di $v$. Deve generalizzare il concetto di modulo in $\CC$ 
$\abs{z} = \sqrt{z^* z}$. Una \emph{norma}, per essere tale deve soddisfare le
le seguenti proprietà:
\begin{enumerate}[1)]
	\item $\norm{v} \geq 0$ e = si verifica $\iff v = 0$
	\item $\norm{\lambda v} = \abs{\lambda} \norm{v}$
	\item $\|v_1 + v_2\| \leq \|v_1\| + \|v_2\|$ (disuguaglianza triangolare)
\end{enumerate}
In questo caso $\left(V, \norm{\;} \right)$ è uno \emph{spazio normato}.
\end{definition}
La norma introduce anche una \emph{metrica}, cioè una distanza fra due elementi
$v_1, v_2$ definita da $d(v_1, v_2) = \|v_1 - v_2\|$. 
\begin{definition}{(distanza e spazi metrici)}
In genere uno \emph{spazio metrico} è un insieme, non necessariamente uno
spazio vettoriale, in cui per ogni coppia di di elementi è definita una
\emph{distanza} $d: V \times V \mapsto \RR$ tale che:
\begin{enumerate}[1)]
	\item $d(v_1, v_2) \geq 0$ e = si verifica $\iff v_1 = v_2$
	\item $d(v_1, v_2) = d(v_2, v_1)$
	\item $d(v_1, v_2) \leq d(v_1, v_3) + d(v_3, v_2)$
\end{enumerate}
\end{definition}
Uno spazio normato $\left(V, \norm{\;} \right)$ è quindi anche uno spazio metrico.
\begin{align*}
	&d(v_1, v_2) = \|v_1 - v_2\| = \|v_2 - v_1\| = d(v_2, v_1) ;\\ 
	&d(v_1, v_2) = \|v_1 - v_2\| = \|v_1 - v_3 +v_3 -v_2 \| \leq
	\|v_1 - v_3\| + \|v_3 - v_2\| = d(v_1, v_3) + d(v_3, v_2).
\end{align*}
Noi ci occuperemo sempre di spazi la cui metrica discende da una norma.
Abbiamo già visto esempi di spazi normati in dimensione finita. Su $\CC^n$,
$z = \left( z_1, z_2, \ldots, z_n \right)$, possiamo definire
$\|z\| = \max_i \abs{z_i}$, per cui 1) e 2) sono ovvie, 3) discende da \[
\|z + w\| = \max_i \abs{z_i + w_i} \leq \max_i \left(\abs{z_i} + \abs{w_i}\right)
\leq \max_i \abs{z_i} + \max_i \abs{w_i} = \|z\| + \|w\|
.\] 
Oppure sempre su $\CC^n$ possiamo prendere $\|z\| = \sum_{i=1}^{n} \abs{z_i}$.
Ancora 1) e 2) sono ovvie, mentre 3):
\[
	\|z + w\| = \sum_{i=1}^{n} \abs{z_i + w_i} \leq \sum_{i=1}^{n} \abs{z_i} +
	\sum_{i=1}^{n} \abs{w_i} = \|z\| + \|w\|
.\] 
Vediamo ora alcuni esempi che generalizzano i casi sopra in dimensione
infinita. Prendiamo le sequenze $z_i$ con $i \in \NN^+$ di numeri complessi.
Quindi $z = \left( z_1, z_2, \ldots, z_n \ldots,  \right)$ limitiamoci alle
sequenze limitate, cioè per cui $\ds \sup_i \abs{z_i} < \infty$. Questo è uno
spazio vettoriale e $\|z\| = \ds \sup_i \abs{z_i}$ è una norma.
Come prima si vede che
\[
	\|z + w \| = \sup_i \abs{z_i + w_i} \leq \sup_i \left(\abs{z_i} + \abs{w_i}
	\right) \leq \sup_i \abs{z_i} + \sup_i \abs{w_i} = \| z + w\| 
.\] 
Possiamo prendere invece le sequenze tali che 
$\sum_i \abs{z_i} < \infty$, e usare come norma proprio 
$\|z\| = \sum_{i} \abs{z_i}$. 

Vediamo altri esempi usando spazi funzionali. Prendiamo l'insieme delle
funzioni continue su un intervallo chiuso $f \in C\left( [a, b] \right)$,
$f: [a, b] \to \CC$. Questo è uno spazio vettoriale e su di esso possiamo
prendere, ad esempio: $\|f\| = \ds \max_{x \in [a, b]} \abs{f(x)}$,
oppure $\|f\| = \int_a^b \abs{f(x)} \ud x$.
Consideriamo ora gli aspetti topologici, cioè le nozioni di limite e continuità
che la norma induce sugli spazi.
\begin{definition}[punto di accumulazione]
Prendiamo un sottoinsieme $A \subset V$, un $\underline{v} \in V$ si dice
\emph{punto di accumulazione} di $A$ se $\forall \eps > 0 \; \exists v \neq
\underline{v}$ con $v \in A$ tale che  $\|v - \underline{v}\| < \eps$. 
\end{definition}
\begin{remark}
$\underline{v}$ può appartenere o meno ad $A$.
\end{remark}
\begin{definition}[chiusura di un insieme]
Si denota con $\overline{A}$ la \emph{chiusura} di $A$, cioè l'unione di $A$ e
di tutti i suoi punti di accumulazione, per cui vale $A \subset \overline{A}$.
\end{definition}
Prendiamo ora una funzione $f: A \subset V \to W$ dove $\left(V, \| \; \| 
\right)$ e $\left(W, \| \; \| \right)$ sono entrambi dotati di norma e $A$ è
il dominio della funzione $f$.
\begin{definition}[limite di una funzione]
Prendiamo un $\underline{v}$ punto di accumulazione di $A$, si dice che 
$\ds \lim_{v \to \underline{v}} f(v) = w \in W$ se 
$\forall \delta > 0 \; \exists \eps > 0$ tale che $ \|v - \underline{v}\| <
\eps$ e $v \neq \underline{v} \Rightarrow \|f(v) - w\| < \delta$.
\end{definition}
\begin{definition}[continuità puntuale]
Nel caso in cui $\underline{v} \in A$ e $\ds \lim_{v \to 
\underline{v}} = w = f(\underline{v})$ allora si dice che la funzione $f$ è
\emph{continua in} $\underline{v}$.
\end{definition}
\begin{definition}[continuità]
Una funzione si dice \emph{continua} se è continua in ogni punto del proprio
dominio.
\end{definition}
\begin{theorem}[unicità del limite]
Il limite di una funzione, se esiste, è unico.
\end{theorem}
\begin{proof}
Supponiamo per assurdo $\ds \lim_{v \to \underline{v}} f(v) = w_1$
e $\ds \lim_{v \to \underline{v}} f(v) = w_2$ con $w_1 \neq w_2$.
Per la 1) $\|w_1 - w_2\| = \alpha > 0$. Prendiamo $\delta < \frac{\alpha}{2}$
abbiamo che $\exists \eps > 0$ tale che $\|v - \underline{v}\| < \eps
\Rightarrow \|f(v) - w_1\| < \frac{\alpha}{2}$ e $\|f(v) - w_2\| <
\frac{\alpha}{2}$. Ora usando la 3):
\[
	\alpha = \|w_1 - w_2\| = \|w_1 - f(v) + f(v) - w_2\| \leq \|w_1 - f(v)\| +
	\|f(v) - w_2\| < \alpha
.\] 
Quindi $\alpha < \alpha$ il che non è possibile.
\end{proof}
Vediamo ora le successioni di elementi $v_i \in V$ con $i = 1,2, \ldots$
\begin{definition}[limite di una successione]
Si dice che $\ds \lim_{i \to \infty} v_i = \utld{v}$
se $\forall \delta > 0 \; \exists N$ tale che, se $i > N \Rightarrow
\| \utld{v} - v_i\| < \delta$.
\end{definition}
\begin{theorem}[esistenza e unicità del limite]
Sia $\utld{v}$ un punto di accumulazione di $A \subset V$,
possiamo sempre trovare una successione $v_i \in A$ di elementi che ha
$\ds \lim_{i \to \infty} v_i = \utld{v}$.
Il limite di una successione, se esiste, è unico.
\end{theorem}
\begin{proof}
Come prima per il limite di una funzione.	
\end{proof}
\begin{definition}[successione di Cauchy]
Una successione si dice \emph{di Cauchy} se $\forall \delta > 0 \; \exists N$
tale che $\forall i, j > N \quad \|v_i - v_j\| < \delta$.
\end{definition}
\begin{definition}[completezza]
Uno spazio si dice \emph{completo} se per ogni successione di Cauchy esiste un
$\underline{v}$ tale che $v_i$ converge a $\underline{v}$.
\end{definition}
\begin{definition}[spazio di Banach]
Uno spazio normato e completo si chiama \emph{spazio di Banach}.
\end{definition}

Vediamone alcuni esempi. Prendiamo $\CC^n$ con la norma $\ds \max_i
\abs{z_i}$ e la successione $z^j =  \left(z_1^j, z_2^j, \ldots, z_n^j \right)$
dove $j = 1, 2, \ldots, \infty$ di Cauchy. Si vede che se $z^j$ è di Cauchy,
le $n$ singole successioni in $\CC$ $z_i^j$ sono di Cauchy e, per la
completezza di $\CC$ convergono $z_i^j \to w_i$.
Ora prendiamo $w = \left(w_1, w_2, \ldots, w_n \right)$ e verifichiamo che
$\ds \lim_{j \to \infty} z^j = w$.

Quindi dalla completezza di $\CC$ discende la completezza di $\CC^n$ con
la norma vista sopra. Vedremo che su $\CC^n$ questo vale per qualsiasi norma.
\begin{example}[norma uniforme]
Prendiamo lo spazio $C([a, b])$ con la norma 
$\|f\| = \ds \max_{x \in [a, b]} \abs{f(x)}$.
Questa è la norma della convergenza uniforme. Se $f_n$ è di Cauchy lo sono
anche tutte le successioni $f_n(x)$ e quindi abbiamo una funzione $f(x)$ a
cui $f_n(x)$ convergono puntualmente $\ds \lim_{n \to \infty} f_n(x) = f(x)\;
\forall x$.

Ma la convergenza è anche uniforme, non solo puntuale, infatti 
$\delta' < \delta \;$, $i,j>N$ $\|f_i-f_j\|<\delta'$ mandando $f\to 0$
abbiamo per $i>N$ $\|f_i-f\|\leq \delta'<\delta$. Siccome la convergenza è
uniforme $f(x)$ è anch'essa continua, quindi lo spazio $C([a,b])$ con
la norma $\ds \max_{x}\abs{f(x}|$ è completo, cioè uno spazio di Banach.
\end{example}
\begin{example}[spazio non completo]
	
Prendiamo ora sempre $C([a, b])$ ma con $\|f\| = \int_{a}^b \abs{f(x)} \ud x$.
Se prendiamo come esempio di successione
\[
f_n(x)=\begin{cases}
0 \qquad &a\leq x \leq c - 1/n\\
\frac{n}{2}(x - c + 1/n) \qquad &c - 1/n \leq x \leq c + 1/n\\
1 \qquad &c + 1/n \leq x \leq b
\end{cases}
\]
sono tutte funzioni continue. $f_n(x)$ converge puntualmente a
\[
f(x) = \begin{cases}
0 \qquad &a \leq x \leq c\\
\frac{1}{2} \qquad &x = c\\
1 \qquad &c < x \leq b
\end{cases}
\]
dove $f(x)$ chiaramente non appartiene a $C([a, b])$. La funzione converge in norma
\[
\|f(x) - f_n(x)\| = 2\int_{c - 1/n}^c \frac{n}{2} \left(x - c + \frac{1}{n}\right)
\ud x = \frac{1}{2nr} \to 0
\]
per $n \to \infty$. Quindi la successione è di Cauchy, ma non converge ad un elemento di
$C([a, b])$, per cui lo spazio non è completo.
\end{example}

\begin{definition}[norme equivalenti]
Due norme $\|\;\|_1$ e $\|\;\|_2$ sullo stesso spazio $V$ si dicono \emph{equivalenti}
se esistono $k, K > 0$ tali che $\forall v \in V$ 
\[
	k\|v\|_1\leq \|v\|_2\leq K\|v\|_1
.\]
\end{definition}

\begin{lemma}
Se due norme sono equivalenti lo sono anche le nozioni topologiche che da
esse derivano.
\end{lemma}
\begin{proof}
Prendiamo ad esempio una successione $v_n\in V$ tale che
$\lim_{n\to \infty}v_n=v$ con la norma $\|\;\|_1$, allora lo stesso deve
essere valido per la norma $\|\;\|_2$. Per definizione vogliamo che
$\forall \delta>0$ $\exists N$ tale che $\|v-v_i\|_2<\delta$ per $i>N$.
Sappiamo che $\forall \delta/K$ $\exists n$ tale che $\|v-v_i\|_1<\delta/K$.
Ma $\|v-v_i\|_2\leq K\|v-v_i\|_1<\delta$. Per dimostrare l'inverso usiamo
$\|v\|_1<\frac{1}{k}\|v\|_2$.
Con ragionamenti analoghi si vede che tutte le altre nozioni topologiche
(limite, continuità,\ldots) sono equivalenti per norme equivalenti.
\end{proof}
Più in generale abbiamo che in dimensione finita tutte le norme sono
equivalenti, per cui per trovare controesempi di norme non equivalenti
dobbiamo guardare agli spazi infiniti.
\begin{example}[norme non equivalenti]
	
Le norme $\|f\|_1=max|f(x)|$ e $\|f\|_2=\int_{a}^b|f(x)dx$ non possono essere
equivalenti. Ad esempio:
\[
f_n(x)=\begin{cases}
0 \qquad &0 \leq x \leq c - 1/n\\
n(x-c+1/n) \qquad &c - 1/n \leq x \leq c\\
n(-x+c+1/n)\qquad &c \leq x \leq c + 1/n\\
0\qquad &c + 1/n \leq x \leq b
\end{cases}
\]
abbiamo $\|f_n\|_1=1$ $\forall n$ mentre $\|f_n\|_2=1/n\to 0$ per
$n\to \infty$ quindi non può esistere $k$ tale che
$\|f_n\|_1\leq k \|f_n\|_2$ $\forall n$.
\end{example}

Vediamo altre nozioni di topologia.\\
\begin{definition}[Palla aperta]
Dato un elemento $v \in V$, definiamo $B(v,R) \subset V$, la \emph{palla aperta}
di raggio $R$ attorno a $v$ come l'insieme degli elementi $w \in V$ tali
che $\|v-w\| < R$. 
\end{definition}

\begin{definition}[Palla chiusa]
$\overline{B(v,R)}$ è la \emph{palla chiusa} data dagli elementi
per cui $\|v-w\| \leq R$.
\end{definition}

\begin{definition}[aperto]
Un insieme $A \subset V$ è \emph{aperto} se per ogni suo elemento $v\in A$
esiste un $\delta$ tale che l'intera palla aperta $B(v, \delta)$ è contenuta
in $A$.
\end{definition}

\begin{definition}[chiuso]
Un insieme $C \subset V$ è \emph{chiuso} se il suo complementare è aperto.
\end{definition}

\begin{theorem}
La palla aperta è un insieme aperto.
\end{theorem}
\begin{proof}
Dato $w$ con $\|w-v\|=r<R$, prendiamo
$\delta < R-r$, se $t \in B(w, \delta)$, $\|t-w\| < \delta$. Quindi $\|t-v\|
\leq \|z-w\| + \|w-v\| = \delta +r < R$ quindi $t \in B(v,R)$. 
\end{proof}
In maniera analoga si dimostra che $\overline{B(v,R)}$ è chiuso.
\begin{theorem}
Un insieme è chiuso $\iff$ contiene tutti i suoi punti di accumulazione:
$C = \overline{C}$.
\end{theorem}
\begin{proof}
Se $v \notin C$ allora $v \in C^c$ cioè $v$ appartiene al
complementare. Essendo $C^c$ aperto esiste una palla attorno a $v$ in cui
nessun elemento appartiene a $C$, quindi $v$ non può essere un punto di
accumulazione di $C$.
\end{proof}
Se $A_i$ sono insiemi aperti, $\bigcup_i A_i = A$ è ancora aperto. Nota che
il numero degli $A_i$ può anche essere infinito numerabile. 
$\bigcap_{i = 1}^n A_i = A$ è aperto. Per l'intersezione potrebbero esserci
casi in cui un'intersezione di infiniti insiemi aperti non produce un aperto.
Dalle formule di De Morgan $(\bigcup_i A_i)^c = \bigcap_i A_i^c$ si deducono
formule analoghe per i chiusi; ora è l'intersezione che può essere infinita
numerabile. $\bigcap_i C_i = C$ è chiuso se $C_i$ sono chiusi; 
$\cup_{i=1}^n C_i = C$ è chiuso se lo sono i $C_i$.

\begin{definition}[compatto]
Un insieme $W$ è \emph{compatto} se per ogni successione $v_i\in W$ si può
estrarre una sottosuccessione $v_{i_{(j)}}$ che converge ad un punto di $W$,
$\lim_{j \to \infty} v_{i_{(j)}} = w \in W$.
\end{definition}

Un insieme compatto contiene tutti i suoi punti di accumulazione ed
è necessariamente chiuso.

\begin{definition}[limitato]
Un insieme è \emph{limitato} se esiste $k$ tale che $\|v\| < k$ per ogni
suo elemento.
\end{definition}

\begin{lemma}[compatto $\implies$ limitato]
Un insieme compatto è necessariamente limitato.
\end{lemma}
\begin{proof}
Se non lo fosse potremmo avere una successione tale che $\|v_n\| > n$, non
può quindi esistere un punto di accumulazione dei $\|v_n\|$ e quindi nemmeno
una sottosuccessione dei $v_n$ convergente.
\end{proof}
\begin{corollary}
Un insieme chiuso contenuto in un compatto è necessariamente compatto.
\end{corollary}

\section{Teorema di Weierstrass}
Vediamo ora l'importante Teorema di Weierstrass:
\begin{theorem}[di Weierstrass]
Sia $f: C \to W$ con $C \subset V$ compatto. Se $f$ è continua, allora
$\|f\|$ assume un massimo e un minimo in $C$.
\end{theorem}
\begin{proof}
Sia $\ds M = \sup_{v \in C} \|f(v)\|$. Per definizione di sup esiste una
successione $v_i$: $i = 1, \ldots, \infty$ tale che 
$ \ds \lim_{i \to \infty} \|f(v_i)\| = M$. Ma per la compattezza esiste
una sottosuccessione $v_{i_{(j)}}$ tale che 
$\ds \lim_{j \to \infty} v_{i_{(j)}} = w$ con $w \in C$.
Per la continuità $\ds \lim_{j \to \infty} f(v_{i_{(j)}}) = f(w)$ e quindi
deve essere $ \ds \|f(w)\| = \lim_{j \to \infty} \|f(v_{i_{(j)}})\| = M$
per la continuità della funzione norma. Quindi il sup è anche un max. 
\end{proof}
È facile trovare esempi in cui l'assenza di compattezza e/o continuità
invalidano l'esistenza del max (o del min).
Abbiamo usato la continuità della funzione norma.
\begin{proof}
Per verificarlo basta prendere $\|v\|=\|v-v_i+v_i\| \leq \|v-v_i\|+\|v_i\|$
insieme a $\|v_i\| = \|v_i - v + v\| \leq \|v - v_i\| + \|v\|$ quindi
$\abs{\|v\| - \|v_i\|} \leq \|v-v_i\|$. Per cui se 
$\ds \lim_{v \to v_i} v_i = v$ si ha anche
$\ds \lim_{v \to v_i} \|v_i\| = \|v\|$.
\end{proof}

\begin{definition}[L-lipschitzianità]
In genere una funzione $f:V \to W$ è detta \emph{L-lipschitziana} se esiste
un $L$ tale che $\|f(v) - f(w)\| \leq L\|v-w\| \quad \forall$ coppia $v,w$.
\end{definition}

Le funzioni lipschitziane sono una sottoclasse di quelle continue.
\begin{example}[lipschitziana]
La funzione norma $\| \; \|: V \to \RR$ è L-lipschitziana con $L = 1$.
\end{example}

Possiamo ora enunciare il seguente
\begin{theorem}[equivalenza delle norme]
In spazi a dimensione finita tutte le norme sono equivalenti sono equivalenti
e quindi inducono la stessa topologia.
\end{theorem}
\begin{proof}
Prendiamo una base $e_1, \ldots, e_n$ che manda lo spazio in $\CC^n$.
Poi prendiamo una norma di riferimento, ad esempio 
$\ds \|z\|_1 = \max_i \abs{z_i}$. Dimostreremo che ogni altra norma $\| \; \|_2$ 
è equivalente a $\| \; \|_1$. $\overline{B(0, 1)} = 
\left\{v: \|v\|_1 \leq 1 \right\} $ è chiuso, 
$\overline{{B(0, 1)}^c} = \left\{v: \|v\|_1 \geq 1 \right\} $ è pure chiuso,
quindi
\[
S(0, 1) := \left\{v: \|v\|_1 = 1\right\} = 
\overline{B(0, 1)} \cap \overline{{B(0,1)}^c}
.\] 
è pure chiuso perché intersezione di chiusi.
Un insieme chiuso e limitato in $\CC^n$ con la norma $\| \; \|_1$ è anche
compatto. Se $z^i = \left(z_1^i, \ldots, z_n^i \right)$ è una successione
limitata lo sono anche tutte le $z_j^i$ in $\CC$. Quindi per ciascuna possiamo
estrarre una sottosuccessione convergente $z_1^{i_{(j)}} \to w_1$ ad un
elemento dell'insieme per $j \to \infty$. Quindi estraiamo una
sottosuccessione per cui converge anche $z_2^{i_{(j_{(t)})}} \to w_2$ e così
via \ldots. Alla fine abbiamo una sottosuccessione 
$z^{i_{(j)}} \to w = \left(w_1, \ldots, w_n \right)$. Convergendo 
$z_k^{i_{(j)}} \to w_k$ singolarmente, converge anche $z^{i_{(j)}} \to w$
nella norma $\| \; \|_1$.
Ora dobbiamo mostrare che ogni norma $\| \; \|_2$ è lipschitziana, e quindi
continua, rispetto a $\| \; \|_1$.
\begin{align*}
\|z\|_2 &= \|z_1e_1 + z_2e_2 + \ldots + z_ne_n\|_2 
\leq \abs{z_1}\|e_1\|_2 + \abs{z_2}\|e_2\|_2 + \ldots + \abs{z_n}\|e_n\|_2 \\
&\leq n \max_i \left(\abs{z_1}\|e_1\|_2 \right) 
\leq n \max_i \abs{z_i} \max_i \|e_i\|_2 = n \|z\|_1 \max_i \|e_i\|_2 
\end{align*}
Quindi qualsiasi norma $\| \; \|_2$ è $K$-lipschitziana con 
$\ds K = n \max_i \|e_i\|_2$.
Restringiamo $\| \; \|_2 : S(0, 1) \to \RR$, questa è una funzione continua
su un compatto, dunque assume un massimo e un minimo $M$ e $m \; > 0$.
Ora prendiamo un $v$ generico e $\utld{v} = \frac{v}{\|v\|}$, dove 
$\|\utld{v}\| = 1$,\\
$\|v\|_2 = \| \|v\|_1 \; \utld{v} \| = \|v\|_1 \|\utld{v}\|_2 \;$ siccome
$\utld{v} \in S(0, 1)$:
\[
m \|v\|_1 \leq \|v\|_2 \leq M \|v\|_1
.\] 
Quindi $\| \; \|_2$ e $\| \; \|_1$ sono norme equivalenti.
\end{proof}
\begin{corollary}
In particolare, visto che in dimensione finita tutte le norme sono
equivalenti, un chiuso e limitato è sempre compatto (se lo è per 
$\| \; \|_1$ lo è per tutte).
\end{corollary}

Inoltre, $\CC^n$, con qualsiasi norma è sempre uno spazio completo.\\
Se per due norme vale $\| \; \|_2\leq K\| \; \|_1$, ma non 
$k\| \; \|_1\leq \| \; \|_2$ allora non sono equivalenti, ma comunque alcune
relazioni si possono stabilire.
Ad esempio se un limite vale per $\| \; \|_1$, allora esso vale anche
per $\| \; \|_2$, ma non è detto il contrario.
Se $A$ è aperto per $\| \; \|_1$ lo è anche per $\| \; \|_2$ ma non
è detto il contrario.
Prendiamo l'esempio di prima di $C([a,b])$ con $\|f\|_1=max_{x\in[a,b]}|f(x)|$,
$\|f\|_2=\int_{a}^b|f(x)|dx$. Abbiamo visto prima che $\| \; \|_1$ non può
essere maggiorata da $\| \; \|_2$. Abbiamo però che $\| \; \|_2$ può
essere maggiorata da $\| \; \|_1$:
\[
	\|f\|_2 = \int_{a}^b \abs{f(x)} dx \leq
	(b-a) \max_{x \in [a, b]} \abs{f(x)} = (b-a) \|f\|_1
.\]
\section{Completamento di uno spazio}
Dato uno spazio $(V, \| \; \|)$ possiamo formalmente costruire il suo
completamento $(\widetilde{V},\| \; \|)$ con un procedimento analogo a quello
che si usa per passare da $Q$ a $\RR$.
\begin{definition}[completamento di un insieme]
Innanzitutto il completamento è definito come uno spazio di Banach tale che
esiste una funzione iniettiva che mappa $V \to \widetilde{V}$ in un sottospazio
di $\widetilde{V}$ che conserva la struttura vettoriale e la norma.
\end{definition}
Possiamo quindi pensare a $\widetilde{V}$ come $V$ con l'aggiunta di tutti
i suoi punti di accumulazione.
\begin{definition}[equivalenza di successioni di Cauchy]
Prendiamo l'insieme di tutte le successioni di Cauchy $\{v_i\}$ in $V$,
e dichiariamo due successioni $\{v_i\} \sim \{w_i\}$ equivalenti se
$\ds \lim_{i \to \infty} \|v_i - w_j\| = 0$.
\end{definition}
Le classi di equivalenza sono lo spazio $\widetilde{V}$ cercato.
Chiamiamo $[\{v_i\}]$ la classe di equivalenza di $\{v_i\}$.
Vediamo che $\widetilde{V}$ è uno spazio vettoriale definendo:
\begin{align*}
\lambda [\{v_i\}] &= \left[\{\lambda v_i\}\right] \quad \text{e} \\
[\{v_i\} + [\{w_i\}] &= [\{v_i + w_i\}] 
\end{align*}
Le operazioni sono ben definite, cioè non dipendono dal rappresentante
nella classe di equivalenza.

Se $\ds \lim_{i \to \infty} \|v_i - v'_i\| = 0 \implies
\lim_{i \to \infty} \|\lambda v_i - \lambda v_j\| = 0$, \\
e se $\ds \lim_{i \to \infty} \|w_i - w'_i\| = 0 \implies
\lim_{i \to \infty} \|v_1 + w_i - \left( v'_i + w'_i \right) \| = 0$.\\
Esiste una naturale immersione di $V$ in $\widetilde{V}$, ad ogni $v\in V$
associamo la successione di Cauchy $\{v_i\}$ costante $v_i = v \; \forall i$.
Definiamo quindi una norma in $\widetilde{V}$: 
$ \ds \|\{v_i\}\| = \lim_{i \to \infty} \|v_i\|$. Si vede che è ben definita,
se
\[
\lim_{i \to \infty} \|v_i - v'_i\| = 0 \implies 
\|[\{v_i\}]\| - \|[\{v_i'\}]\| = \lim_{i \to \infty} (\|v_i\| - \|v'_i\|) = 0
.\] 
In particolare l'immersione di $V$ in $\widetilde{V}$ con la successione
costante conserva la norma.\\
Ora vediamo che $V$ è denso in $\widetilde{V}$:
\begin{proof}
Prendiamo una generica $[\{v_i\}]$, poi prendiamo una successione di
successioni $[\{v_j\}]$ costanti, cioè:
\[
[\{v_1,v_1,...,v_1,..\}],\; [\{v_2,v_2,...,v_2,\}], \ldots
.\]
Vediamo che questa successione converge a
$[\{v_1, v_2, \ldots, v_k, \ldots\}]$.\\
$ \ds \|[\{v_i\}] - [\{v_j\}]\| = \lim_{i \to \infty} \|v_i - v_j\|$,
ma siccome $\{v_i\}$ è di Cauchy $\forall \eps > 0$, $\exists N$
tale che per $j > N$ $\ds \lim_{i \to \infty} \|v_i - v_j\| < \eps \;$,
quindi $[\{v_j\}] \to [\{v_i\}]$.
Allo stesso modo si vede che $\hat{V}$ è il completamento di $V$, data la
successione $[\{v_j\}]$ costruiamo $[\{v_1, v_2, \ldots, v_n, \ldots\}]$
a cui essa converge.
\end{proof}

$\widetilde{V}$ è uno spazio completo. Prendiamo una successione di Cauchy di
successioni di Cauchy $[\{v_i^j\}]$. Visto che $V$ è denso in $\widetilde{V}$
possiamo trovare successioni costanti arbitrariamente vicine a quelle sopra.
Per ogni classe di equivalenza di successioni riusciamo ad ottenere degli
elementi tutti in $V$, il cui limite è uguale all'elemento $v_i$.
Prendiamo ad esempio $[\{w^j\}]$ (la successione costante
$\{w^1, w^1, \ldots, w^1\}]$) tale che
$\| [\{w^j\}] - [\{v_i^j\}] \| < \frac{1}{j}$. Abbiamo che 
\[
\|[\{w^j\}] - [\{w^k\}]\| \leq
\frac{1}{j} + \frac{1}{k} + \|[\{v_i^0\}] - [\{v_i^k\}]\|
.\]
quindi anche $[\{w^1\}]$ è di Cauchy e converge a 
$[\{w^i\}] = \{w^1, w^2, w^3, \ldots\}]$ siccome
\[
\|[\{v_i^j\}] - [\{w^i\}]\| \leq \frac{1}{j} + \|[\{w^0\}] - [\{w^i\}]\|
.\]
abbiamo che anche $[\{v_i^j\}] \to [\{w^i\}]$ per $j \to \infty$.

Vediamo altri esempi ed esercizi sulle norme.\\
Prendiamo le funzioni $f: [-1, 1] \to \CC$ continue e derivabili con
derivata continua e quindi limitata su $[-1, 1]$.
Prendiamo $ \ds \|f\|_1 = \max_{x \in [-1, 1]} \abs{f(x)}$ e $ \ds
\|f\|_2 = \max_{x \in [-1, 1]} \abs{f(x)} + \max_{x \in [-1, 1]} \abs{f'(x)}$.
Abbiamo $\|f\|_1 \leq \|f\|_2 \; \forall f$, quindi $\| \; \|_1$ è maggiorata
da $\| \; \|_2$ con $k = 1$.\\
$\| \; \|_2$ però non può essere maggiorata da $\| \; \|_1$.
Ad esempio prendiamo $f_n = x^n$, $\|f_n\|_1 = 1$ e $\|f_n\|_2= 1 + n$.
Le due norme non sono equivalenti: prendiamo $g_n=|x|^{1+1/n}$,
$g_n\in C^1([-1, 1]) \subset C^0([-1, 1])$. $g_n \to g = |x|$ 
per $n \to \infty$ nella norma $\| \; \|_1$, infatti
\[
\|g_n - g\| = \max_{x \in [0,1] }(x(1 - x^{1/n})) = \frac{n^n}
{(n+1)^{n+1}} \to 0
.\]
$g \notin C^1([-1, 1])$ perché non derivabile in $0$.

\begin{definition}[norma $p$]
In $\CC^n$ posiamo introdurre la \emph{norma} $p$:
\[
\|z\|_p=\left(\sum_{i=1}^n|z_i|^p\right)^{1/p}
\]
\end{definition}
Per $p=1, 2$ si riconduce ai casi già considerati 
$\ds \|z\|_1 = \sum_i \abs{z_i}$, $\ds \|z\|_2 = \sqrt{ \sum_i \abs{z_i}^2 }$.
Verifichiamo che $\| \; \|_p$ è una norma per tutti i $p \geq 1$.
Le proprietà $1)$ e $2)$ sono soddisfatte. Ci serve la seguente
disuguaglianza, che generalizza quella di Cauchy-Schwartz:
\begin{theorem}[disuguaglianza di H\"{o}lder]
\[
\sum_i|z_i^*w_i|\leq \left(\sum_i|z_i|^p\right)^{1/p}\left(\sum_i|w_i|^q\right)^{1/q}
\]
$\forall p \geq 1$ con $\dfrac{1}{p} + \dfrac{1}{q} = 1$
\end{theorem}
(Se $p = q = 2$ è Schwartz).
\begin{proof}
Chiamiamo
$\ds \xi_i = \frac{\abs{z_i}^p}{\sum_j \abs{z_j}^p}$
e
$\ds \eta_i = \frac{\abs{w_i}^q}{\sum_j \abs{w_j}^q}$
la disuguaglianza si riscrive come
$\ds \sum_i \eta_i^{1/q} \xi_i^{1/p} \leq 1$ dove 
$\ds \sum_i \xi_i = \sum_i \eta_i = 1$.\\
Usiamo che per $\alpha \leq 1$  si ha $t^\alpha \leq \alpha t + 1 - \alpha$,
$(\alpha t + 1 - \alpha - t^\alpha)' = \alpha - \alpha t^{\alpha-1}$
quindi il minimo di $\alpha t + 1 - \alpha - t^\alpha$ si ha
per $t=1$ ed è uguale a zero.\\
Chiamiamo $t = \xi_i/\eta_i$ e $\alpha = 1/p$, quindi dev'essere
$1 - \alpha = 1/q$ e abbiamo
\[
\left( \frac{\xi_i}{\eta_i} \right)^{1/p} \leq 
\frac{1}{p} \frac{\xi_i}{\eta_i} + \frac{1}{q} \implies
\xi_i^{1/p} \eta_i^{1/q} \leq \frac{1}{p} \xi_i + \frac{1}{q} \eta_i
.\]
Sommando su $i, \;$ $\ds \sum_i \eta_i^{1/q} \xi_i^{1/p} \leq 1$
che è la disuguaglianza cercata.
\end{proof}
La disuguaglianza è saturata solo se $\xi_i = \eta_i \;$ $\forall i$. \\
Possiamo scrivere la disuguaglianza di H\"{o}lder come 
\[
(z, w) \leq \|z\|_p \|w\|_q, \qquad \frac{1}{p} + \frac{1}{q} = 1
\]
La disuguaglianza triangolare per $\| \; \|_p$ è chiamata
\begin{theorem}[disuguaglianza di Minkowski]
\[
\|z+w\|_p\leq \|z\|_p+\|w\|_p
.\]
\end{theorem}
\begin{proof}
partiamo da
\[\sum_i(|z_i|+|w_i|)^p=\sum_i|z_i|(|z_i|+|w_i|)^{p-1}+\sum_i|w_i|(|z_i|+|w_i|)^{p-1}\]
ora usiamo H\"{o}lder per entrambi i termini
\[\leq\left(\left(\sum_i|z_i|^p\right)^{1/p}+\left(\sum_i|w_i|^p\right)^{1/p}\right)\left(\sum_i(|z_i|+|w_i|)^{q(p-1)}\right)^{1/q}\]
\[=(\|z\|_p+\|w\|_p)\left(\sum_i(|z_i|+|w_i|)^p\right)^{\frac{p-1}{p}}\hspace{8mm} q=\frac{1}{1-1/p}=\frac{p}{p-1}\]
quindi otteniamo
\[\left(\sum_i(|z_i|+|w_i|)^p\right)^{1/p}\leq \|z\|_p+\|w\|_p\]
infine
\[
\|z+w\|_p=\left(\sum_i|z_i+w_i|^p\right)^{1/p}\leq \left(\sum_i(|z_i|+|w_i|)\right)^{1/p}\leq |z\|_p+\|w\|_p
.\]
\end{proof}

Se prendiamo successioni infinite $z = \{z_1, z_2, \ldots \}$ possiamo
introdurre le norme\\
$\ds \|z\|_p = \left(\sum_{i=1}^\infty \abs{z_i}^p\right)^{1/p}$.
Occorre però scegliere uno spazio adeguato. 

\begin{definition}[spazi $\ell^p$]
Si definisce lo spazio $\ell^p$ l'insieme delle successioni per cui
$\sum_i^{\infty} \abs{z_i}^p < \infty$ e quindi definiamo $\|z\|_p$ come sopra.
\end{definition}
Le proprietà $1)$, $2)$, e $3)$ della norma valgono come per le $n$-ple
finite.
Vedremo che $\ell^p$ è uno spazio normato completo.
Prendiamo in $\CC^2$ $\|z\|_p = (\abs{z_1}^p + \abs{z_2}^p)^{1/p}$,
con $z_1 = \rho_1 e^{i \theta_1}$ e $z_2 = \rho_2 e^{i \theta_2}$. Vediamo
che le norme $\| \; \|_p$ non dipendono dalle fasi $\theta_1$ e $\theta_2$:
$\|z\|_p = (\rho_1^p + \rho_2^p)^{1/p}$, $S_p(0,1) = \{z:\|z\|_p=1\}$
\[
(\rho_1^p + \rho_2^p)^{1/p} = 1 \implies \rho_2 = (1 - \rho_1^p)^{1/p}
\]
notiamo che, per $\rho_1 < 1, \;$ 
$\ds \lim_{p \to \infty} (1 - \rho_1^p)^{1/p} = 1$.
\[
\lim_{p \to \infty} S_p(0, 1) = \{(\rho_1, \rho_2): \max \{\rho_1, \rho_2\} = 1
	:= S_{\max}(0, 1)
\]
quindi $\| \; \|_p$ per $p \to \infty$ diventa
$\| \; \|_{\infty} = max_i \abs{z_i}$
(questo vale sia per $\CC^n$ che per le successioni infinite).\\
Calcoliamo $m$ e $M$ che abbiamo usato per dimostrare la equivalenza
delle norme. La norma di riferimento è $\| \; \|_{\infty}$.
Chiamiamo $m_p$ e $M_p$ il minimo e il massimo assunto da $\| \; \|_p$
nell'insieme $\|z\|_{\infty} = \max \{\rho_1, \rho_2\}=1$.
Prendiamo $\rho_1 = 1$ e $0 \leq \rho_2 \leq 1$ (l'altro pezzo è equivalente)
$\|z\|_p = (1 + \rho_2^p)^{1/p}$
quindi il minimo è realizzato in $\rho_1 = 0$ e il massimo in $\rho_2 = 1$,
$m_p = 1$, $M_p = 2^{1/p}$.
In $\CC^n$ possiamo sempre ricondurci ai moduli degli 
$z_i = \rho_i e^{i \theta_i}$
\[
\|z\|_p = \left(\sum_i \rho_i^p\right)^{1/p}, \qquad S_{\infty}(0, 1) =
\bigcup_i \{\rho_i = 1\}
\]
\begin{align*}
m_p &= \min_{p_i > 1} (1 + \sum_{i>1} \rho_i^p)^{1/p} = 1 \\
M_p &= \max_{\rho_i > 1} \left(1 + \sum_{i>1} \rho_i^p\right)^{1/p} = n^{1/p}
\end{align*}
