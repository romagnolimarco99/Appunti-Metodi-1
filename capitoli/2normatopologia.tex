\chapter{Norma e Topologia}
Prendiamo uno spazio vettoriale $V$ su $\CC$ di dimensione arbitraria, non
necessariamente finita. La discussione varrà anche per spazi definiti su
$\RR$, basta trascurare *(il coniugio) quando compare. Ora equipaggiamo $V$
con una norma.
\begin{definition}[(norma e spazi normati)]
Una \emph{norma} $\| \|: V \mapsto \RR$ è una funzione che ad ogni elemento
$v \in V$ associa un numero reale $\|v\|$, da intendersi come una nozione di
"lunghezza" di $v$. Deve generalizzare il concetto di modulo in $\CC$ 
$\abs{z} = \sqrt{z^* z}$. Una \emph{norma}, per essere tale deve soddisfare le
le seguenti proprietà:
\begin{enumerate}[1)]
	\item $\norm{v} \geq 0$ e = si verifica $\iff v = 0$
	\item $\norm{\lambda v} = \abs{\lambda} \norm{v}$
	\item $\|v_1 + v_2\| \leq \|v_1\| + \|v_2\|$ (disuguaglianza triangolare)
\end{enumerate}
In questo caso $\left(V, \norm{\cdot} \right)$ è uno \emph{spazio normato}.
\end{definition}
La norma introduce anche una \emph{metrica}, cioè una distanza fra due elementi
$v_1, v_2$ definita da $d(v_1, v_2) = \|v_1 - v_2\|$. 
\begin{definition}{(distanza e spazi metrici)}
In genere uno \emph{spazio metrico} è un insieme, non necessariamente uno
spazio vettoriale, in cui per ogni coppia di di elementi è definita una
\emph{distanza} $d: V \times V \mapsto \RR$ tale che:
\begin{enumerate}[1)]
	\item $d(v_1, v_2) \geq 0$ e = si verifica $\iff v_1 = v_2$
	\item $d(v_1, v_2) = d(v_2, v_1)$
	\item $d(v_1, v_2) \leq d(v_1, v_3) + d(v_3, v_2)$
\end{enumerate}
\end{definition}
Uno spazio normato $\left(V, \norm{\cdot} \right)$ è quindi anche uno spazio metrico.
\begin{align*}
	&d(v_1, v_2) = \|v_1 - v_2\| = \|v_2 - v_1\| = d(v_2, v_1) ;\\ 
	&d(v_1, v_2) = \|v_1 - v_2\| = \|v_1 - v_3 +v_3 -v_2 \| \leq
	\|v_1 - v_3\| + \|v_3 - v_2\| = d(v_1, v_3) + d(v_3, v_2).
\end{align*}
Noi ci occuperemo sempre di spazi la cui metrica discende da una norma.
Abbiamo già visto esempi di spazi normati in dimensione finita. Su $\CC^n$,
$z = \left( z_1, z_2, \ldots, z_n \right)$, possiamo definire
$\|z\| = \max_i \abs{z_i}$, per cui 1) e 2) sono ovvie, 3) discende da \[
\|z + w\| = \max_i \abs{z_i + w_i} \leq \max_i \left(\abs{z_i} + \abs{w_i}\right)
\leq \max_i \abs{z_i} + \max_i \abs{w_i} = \|z\| + \|w\|
.\] 
Oppure sempre su $\CC^n$ possiamo prendere $\|z\| = \sum_{i=1}^{n} \abs{z_i}$.
Ancora 1) e 2) sono ovvie, mentre 3):
\[
	\|z + w\| = \sum_{i=1}^{n} \abs{z_i + w_i} \leq \sum_{i=1}^{n} \abs{z_i} +
	\sum_{i=1}^{n} \abs{w_i} = \|z\| + \|w\|
.\] 
Vediamo ora alcuni esempi che generalizzano i casi sopra in dimensione
infinita. Prendiamo le sequenze $z_i$ con $i \in \NN^+$ di numeri complessi.
Quindi $z = \left( z_1, z_2, \ldots, z_n \ldots,  \right)$ limitiamoci alle
sequenze limitate, cioè per cui $\displaystyle \sup_i \abs{z_i} < \infty$. Questo è uno
spazio vettoriale e $\|z\| = \displaystyle \sup_i \abs{z_i}$ è una norma.
Come prima si vede che
\[
	\|z + w \| = \sup_i \abs{z_i + w_i} \leq \sup_i \left(\abs{z_i} + \abs{w_i}
	\right) \leq \sup_i \abs{z_i} + \sup_i \abs{w_i} = \| z + w\| 
.\] 
Possiamo prendere invece le sequenze tali che 
$\sum_i \abs{z_i} < \infty$, e usare come norma proprio 
$\|z\| = \sum_{i} \abs{z_i}$. 

Vediamo altri esempi usando spazi funzionali. Prendiamo l'insieme delle
funzioni continue su un intervallo chiuso $f \in C\left( [a, b] \right)$,
$f: [a, b] \to \CC$. Questo è uno spazio vettoriale e su di esso possiamo
prendere, ad esempio: $\|f\| = \displaystyle \max_{x \in [a, b]} \abs{f(x)}$,
oppure $\|f\| = \int_a^b \abs{f(x)} \ud x$.
Consideriamo ora gli aspetti topologici, cioè le nozioni di limite e continuità
che la norma induce sugli spazi.
\begin{definition}[punto di accumulazione]
Prendiamo un sottoinsieme $A \subset V$, un $\underline{v} \in V$ si dice
\emph{punto di accumulazione} di $A$ se $\forall \eps > 0 \; \exists v \neq
\underline{v}$ con $v \in A$ tale che  $\|v - \underline{v}\| < \eps$. 
\end{definition}
\begin{remark}
$\underline{v}$ può appartenere o meno ad $A$.
\end{remark}
\begin{definition}[chiusura di un insieme]
Si denota con $\overline{A}$ la \emph{chiusura} di $A$, cioè l'unione di $A$ e
di tutti i suoi punti di accumulazione, per cui vale $A \subset \overline{A}$.
\end{definition}
Prendiamo ora una funzione $f: A \subset V \to W$ dove $\left(V, \| \cdot \| 
\right)$ e $\left(W, \| \cdot \| \right)$ sono entrambi dotati di norma e $A$ è
il dominio della funzione $f$.
\begin{definition}[limite di una funzione]
Prendiamo un $\underline{v}$ punto di accumulazione di $A$, si dice che 
$\displaystyle \lim_{v \to \underline{v}} f(v) = w \in W$ se 
$\forall \delta > 0 \; \exists \eps > 0$ tale che $ \|v - \underline{v}\| <
\eps$ e $v \neq \underline{v} \Rightarrow \|f(v) - w\| < \delta$.
\end{definition}
\begin{definition}[continuità puntuale]
Nel caso in cui $\underline{v} \in A$ e $\displaystyle \lim_{v \to 
\underline{v}} = w = f(\underline{v})$ allora si dice che la funzione $f$ è
\emph{continua in} $\underline{v}$.
\end{definition}
\begin{definition}[continuità]
Una funzione si dice \emph{continua} se è continua in ogni punto del proprio
dominio.
\end{definition}
\begin{theorem}[unicità del limite]
Il limite di una funzione, se esiste, è unico.
\end{theorem}
\begin{proof}
Supponiamo per assurdo $\displaystyle \lim_{v \to \underline{v}} f(v) = w_1$
e $\displaystyle \lim_{v \to \underline{v}} f(v) = w_2$ con $w_1 \neq w_2$.
Per la 1) $\|w_1 - w_2\| = \alpha > 0$. Prendiamo $\delta < \frac{\alpha}{2}$
abbiamo che $\exists \eps > 0$ tale che $\|v - \underline{v}\| < \eps
\Rightarrow \|f(v) - w_1\| < \frac{\alpha}{2}$ e $\|f(v) - w_2\| <
\frac{\alpha}{2}$. Ora usando la 3):
\[
	\alpha = \|w_1 - w_2\| = \|w_1 - f(v) + f(v) - w_2\| \leq \|w_1 - f(v)\| +
	\|f(v) - w_2\| < \alpha
.\] 
Quindi $\alpha < \alpha$ il che non è possibile.
\end{proof}
Vediamo ora le successioni di elementi $v_i \in V$ con $i = 1,2, \ldots$
\begin{definition}[limite di una successione]
Si dice che $\displaystyle \lim_{i \to \infty} v_i = \underaccent{\tilde}{v}$
se $\forall \delta > 0 \; \exists N$ tale che, se $i > N \Rightarrow
\| \underaccent{\tilde}{v} - v_i\| < \delta$.
\end{definition}
\begin{theorem}[esistenza e unicità del limite]
Sia $\underaccent{\tilde}{v}$ un punto di accumulazione di $A \subset V$,
possiamo sempre trovare una successione $v_i \in A$ di elementi che ha
$\displaystyle \lim_{i \to \infty} v_i = \underaccent{\tilde}{v}$.
Il limite di una successione, se esiste, è unico.
\end{theorem}
\begin{proof}
Come prima per il limite di una funzione.	
\end{proof}
\begin{definition}[successione di Cauchy]
Una successione si dice \emph{di Cauchy} se $\forall \delta > 0 \; \exists N$
tale che $\forall i, j > N \quad \|v_i - v_j\| < \delta$.
\end{definition}
\begin{definition}[completezza]
Uno spazio si dice \emph{completo} se per ogni successione di Cauchy esiste un
$\underline{v}$ tale che $v_i$ converge a $\underline{v}$.
\end{definition}
\begin{definition}[spazio di Banach]
Uno spazio normato e completo si chiama \emph{spazio di Banach}.
\end{definition}

Vediamone alcuni esempi. Prendiamo $\CC^n$ con la norma $\displaystyle \max_i
\abs{z_i}$ e la successione $z^j =  \left(z_1^j, z_2^j, \ldots, z_n^j \right)$
dove $j = 1, 2, \ldots, \infty$ di Cauchy. Si vede che se $z^j$ è di Cauchy,
le $n$ singole successioni in $\CC$ $z_i^j$ sono di Cauchy e, per la
completezza di $\CC$ convergono $z_i^j \to w_i$.
Ora prendiamo $w = \left(w_1, w_2, \ldots, w_n \right)$ e verifichiamo che
$\displaystyle \lim_{j \to \infty} z^j = w$. Quindi dalla completezza di $\CC$ 
discende la completezza di $\CC^n$ con la norma vista sopra. Vedremo che su
$\CC^n$ questo vale per qualsiasi norma.
