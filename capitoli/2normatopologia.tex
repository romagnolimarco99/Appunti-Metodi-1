\chapter{Norma e Topologia}
Prendiamo uno spazio vettoriale $V$ su $\CC$ di dimensione arbitraria, non
necessariamente finita. La discussione varrà anche per spazi definiti su
$\RR$, basta trascurare *(il coniugio) quando compare. Ora equipaggiamo $V$
con una norma.
\begin{definition}[(norma e spazi normati)]
Una \emph{norma} $\| \|: V \mapsto \RR$ è una funzione che ad ogni elemento
$v \in V$ associa un numero reale $\|v\|$, da intendersi come una nozione di
"lunghezza" di $v$. Deve generalizzare il concetto di modulo in $\CC$ 
$\abs{z} = \sqrt{z^* z}$. Una \emph{norma}, per essere tale deve soddisfare le
le seguenti proprietà:
\begin{enumerate}[1)]
	\item $\norm{v} \geq 0$ e = si verifica $\iff v = 0$
	\item $\norm{\lambda v} = \abs{\lambda} \norm{v}$
	\item $\|v_1 + v_2\| \leq \|v_1\| + \|v_2\|$ (disuguaglianza triangolare)
\end{enumerate}
In questo caso $\left(V, \norm{\;} \right)$ è uno \emph{spazio normato}.
\end{definition}
La norma introduce anche una \emph{metrica}, cioè una distanza fra due elementi
$v_1, v_2$ definita da $d(v_1, v_2) = \|v_1 - v_2\|$. 
\begin{definition}{(distanza e spazi metrici)}
In genere uno \emph{spazio metrico} è un insieme, non necessariamente uno
spazio vettoriale, in cui per ogni coppia di di elementi è definita una
\emph{distanza} $d: V \times V \mapsto \RR$ tale che:
\begin{enumerate}[1)]
	\item $d(v_1, v_2) \geq 0$ e = si verifica $\iff v_1 = v_2$
	\item $d(v_1, v_2) = d(v_2, v_1)$
	\item $d(v_1, v_2) \leq d(v_1, v_3) + d(v_3, v_2)$
\end{enumerate}
\end{definition}
Uno spazio normato $\left(V, \norm{\;} \right)$ è quindi anche uno spazio metrico.
\begin{align*}
	&d(v_1, v_2) = \|v_1 - v_2\| = \|v_2 - v_1\| = d(v_2, v_1) ;\\ 
	&d(v_1, v_2) = \|v_1 - v_2\| = \|v_1 - v_3 +v_3 -v_2 \| \leq
	\|v_1 - v_3\| + \|v_3 - v_2\| = d(v_1, v_3) + d(v_3, v_2).
\end{align*}
Noi ci occuperemo sempre di spazi la cui metrica discende da una norma.
Abbiamo già visto esempi di spazi normati in dimensione finita. Su $\CC^n$,
$z = \left( z_1, z_2, \ldots, z_n \right)$, possiamo definire
$\|z\| = \max_i \abs{z_i}$, per cui 1) e 2) sono ovvie, 3) discende da \[
\|z + w\| = \max_i \abs{z_i + w_i} \leq \max_i \left(\abs{z_i} + \abs{w_i}\right)
\leq \max_i \abs{z_i} + \max_i \abs{w_i} = \|z\| + \|w\|
.\] 
Oppure sempre su $\CC^n$ possiamo prendere $\|z\| = \sum_{i=1}^{n} \abs{z_i}$.
Ancora 1) e 2) sono ovvie, mentre 3):
\[
	\|z + w\| = \sum_{i=1}^{n} \abs{z_i + w_i} \leq \sum_{i=1}^{n} \abs{z_i} +
	\sum_{i=1}^{n} \abs{w_i} = \|z\| + \|w\|
.\] 
Vediamo ora alcuni esempi che generalizzano i casi sopra in dimensione
infinita. Prendiamo le sequenze $z_i$ con $i \in \NN^+$ di numeri complessi.
Quindi $z = \left( z_1, z_2, \ldots, z_n \ldots,  \right)$ limitiamoci alle
sequenze limitate, cioè per cui $\ds \sup_i \abs{z_i} < \infty$. Questo è uno
spazio vettoriale e $\|z\| = \ds \sup_i \abs{z_i}$ è una norma.
Come prima si vede che
\[
	\|z + w \| = \sup_i \abs{z_i + w_i} \leq \sup_i \left(\abs{z_i} + \abs{w_i}
	\right) \leq \sup_i \abs{z_i} + \sup_i \abs{w_i} = \| z + w\| 
.\] 
Possiamo prendere invece le sequenze tali che 
$\sum_i \abs{z_i} < \infty$, e usare come norma proprio 
$\|z\| = \sum_{i} \abs{z_i}$. 

Vediamo altri esempi usando spazi funzionali. Prendiamo l'insieme delle
funzioni continue su un intervallo chiuso $f \in C\left( [a, b] \right)$,
$f: [a, b] \to \CC$. Questo è uno spazio vettoriale e su di esso possiamo
prendere, ad esempio: $\|f\| = \ds \max_{x \in [a, b]} \abs{f(x)}$,
oppure $\|f\| = \int_a^b \abs{f(x)} \ud x$.
Consideriamo ora gli aspetti topologici, cioè le nozioni di limite e continuità
che la norma induce sugli spazi.
\begin{definition}[punto di accumulazione]
Prendiamo un sottoinsieme $A \subset V$, un $\underline{v} \in V$ si dice
\emph{punto di accumulazione} di $A$ se $\forall \eps > 0 \; \exists v \neq
\underline{v}$ con $v \in A$ tale che  $\|v - \underline{v}\| < \eps$. 
\end{definition}
\begin{remark}
$\underline{v}$ può appartenere o meno ad $A$.
\end{remark}
\begin{definition}[chiusura di un insieme]
Si denota con $\overline{A}$ la \emph{chiusura} di $A$, cioè l'unione di $A$ e
di tutti i suoi punti di accumulazione, per cui vale $A \subset \overline{A}$.
\end{definition}
Prendiamo ora una funzione $f: A \subset V \to W$ dove $\left(V, \| \; \| 
\right)$ e $\left(W, \| \; \| \right)$ sono entrambi dotati di norma e $A$ è
il dominio della funzione $f$.
\begin{definition}[limite di una funzione]
Prendiamo un $\underline{v}$ punto di accumulazione di $A$, si dice che 
$\ds \lim_{v \to \underline{v}} f(v) = w \in W$ se 
$\forall \delta > 0 \; \exists \eps > 0$ tale che $ \|v - \underline{v}\| <
\eps$ e $v \neq \underline{v} \Rightarrow \|f(v) - w\| < \delta$.
\end{definition}
\begin{definition}[continuità puntuale]
Nel caso in cui $\underline{v} \in A$ e $\ds \lim_{v \to 
\underline{v}} = w = f(\underline{v})$ allora si dice che la funzione $f$ è
\emph{continua in} $\underline{v}$.
\end{definition}
\begin{definition}[continuità]
Una funzione si dice \emph{continua} se è continua in ogni punto del proprio
dominio.
\end{definition}
\begin{theorem}[unicità del limite]
Il limite di una funzione, se esiste, è unico.
\end{theorem}
\begin{proof}
Supponiamo per assurdo $\ds \lim_{v \to \underline{v}} f(v) = w_1$
e $\ds \lim_{v \to \underline{v}} f(v) = w_2$ con $w_1 \neq w_2$.
Per la 1) $\|w_1 - w_2\| = \alpha > 0$. Prendiamo $\delta < \frac{\alpha}{2}$
abbiamo che $\exists \eps > 0$ tale che $\|v - \underline{v}\| < \eps
\Rightarrow \|f(v) - w_1\| < \frac{\alpha}{2}$ e $\|f(v) - w_2\| <
\frac{\alpha}{2}$. Ora usando la 3):
\[
	\alpha = \|w_1 - w_2\| = \|w_1 - f(v) + f(v) - w_2\| \leq \|w_1 - f(v)\| +
	\|f(v) - w_2\| < \alpha
.\] 
Quindi $\alpha < \alpha$ il che non è possibile.
\end{proof}
Vediamo ora le successioni di elementi $v_i \in V$ con $i = 1,2, \ldots$
\begin{definition}[limite di una successione]
Si dice che $\ds \lim_{i \to \infty} v_i = \utld{v}$
se $\forall \delta > 0 \; \exists N$ tale che, se $i > N \Rightarrow
\| \utld{v} - v_i\| < \delta$.
\end{definition}
\begin{theorem}[esistenza e unicità del limite]
Sia $\utld{v}$ un punto di accumulazione di $A \subset V$,
possiamo sempre trovare una successione $v_i \in A$ di elementi che ha
$\ds \lim_{i \to \infty} v_i = \utld{v}$.
Il limite di una successione, se esiste, è unico.
\end{theorem}
\begin{proof}
Come prima per il limite di una funzione.	
\end{proof}
\begin{definition}[successione di Cauchy]
Una successione si dice \emph{di Cauchy} se $\forall \delta > 0 \; \exists N$
tale che $\forall i, j > N \quad \|v_i - v_j\| < \delta$.
\end{definition}
\begin{definition}[completezza]
Uno spazio si dice \emph{completo} se per ogni successione di Cauchy esiste un
$\underline{v}$ tale che $v_i$ converge a $\underline{v}$.
\end{definition}
\begin{definition}[spazio di Banach]
Uno spazio normato e completo si chiama \emph{spazio di Banach}.
\end{definition}

Vediamone alcuni esempi. Prendiamo $\CC^n$ con la norma $\ds \max_i
\abs{z_i}$ e la successione $z^j =  \left(z_1^j, z_2^j, \ldots, z_n^j \right)$
dove $j = 1, 2, \ldots, \infty$ di Cauchy. Si vede che se $z^j$ è di Cauchy,
le $n$ singole successioni in $\CC$ $z_i^j$ sono di Cauchy e, per la
completezza di $\CC$ convergono $z_i^j \to w_i$.
Ora prendiamo $w = \left(w_1, w_2, \ldots, w_n \right)$ e verifichiamo che
$\ds \lim_{j \to \infty} z^j = w$.

Quindi dalla completezza di $\CC$ discende la completezza di $\CC^n$ con
la norma vista sopra. Vedremo che su $\CC^n$ questo vale per qualsiasi norma.
\begin{example}[norma uniforme]
Prendiamo lo spazio $C([a, b])$ con la norma 
$\|f\| = \ds \max_{x \in [a, b]} \abs{f(x)}$.
Questa è la norma della convergenza uniforme. Se $f_n$ è di Cauchy lo sono
anche tutte le successioni $f_n(x)$ e quindi abbiamo una funzione $f(x)$ a
cui $f_n(x)$ convergono puntualmente $\ds \lim_{n \to \infty} f_n(x) = f(x)\;
\forall x$.

Ma la convergenza è anche uniforme, non solo puntuale, infatti 
$\delta' < \delta \;$, $i,j>N$ $\|f_i-f_j\|<\delta'$ mandando $f\to 0$
abbiamo per $i>N$ $\|f_i-f\|\leq \delta'<\delta$. Siccome la convergenza è
uniforme $f(x)$ è anch'essa continua, quindi lo spazio $C([a,b])$ con
la norma $\ds \max_{x}\abs{f(x}|$ è completo, cioè uno spazio di Banach.
\end{example}
\begin{example}[spazio non completo]
	
Prendiamo ora sempre $C([a, b])$ ma con $\|f\| = \int_{a}^b \abs{f(x)} \ud x$.
Se prendiamo come esempio di successione
\[
f_n(x)=\begin{cases}
0 \qquad &a\leq x \leq c - 1/n\\
\frac{n}{2}(x - c + 1/n) \qquad &c - 1/n \leq x \leq c + 1/n\\
1 \qquad &c + 1/n \leq x \leq b
\end{cases}
\]
sono tutte funzioni continue. $f_n(x)$ converge puntualmente a
\[
f(x) = \begin{cases}
0 \qquad &a \leq x \leq c\\
\frac{1}{2} \qquad &x = c\\
1 \qquad &c < x \leq b
\end{cases}
\]
dove $f(x)$ chiaramente non appartiene a $C([a, b])$. La funzione converge in norma
\[
\|f(x) - f_n(x)\| = 2\int_{c - 1/n}^c \frac{n}{2} \left(x - c + \frac{1}{n}\right)
\ud x = \frac{1}{2nr} \to 0
\]
per $n \to \infty$. Quindi la successione è di Cauchy, ma non converge ad un elemento di
$C([a, b])$, per cui lo spazio non è completo.
\end{example}

\begin{definition}[norme equivalenti]
Due norme $\|\;\|_1$ e $\|\;\|_2$ sullo stesso spazio $V$ si dicono \emph{equivalenti}
se esistono $k, K > 0$ tali che $\forall v \in V$ 
\[
	k\|v\|_1\leq \|v\|_2\leq K\|v\|_1
.\]
\end{definition}

\begin{lemma}
Se due norme sono equivalenti lo sono anche le nozioni topologiche che da
esse derivano.
\end{lemma}
\begin{proof}
Prendiamo ad esempio una successione $v_n\in V$ tale che
$\lim_{n\to \infty}v_n=v$ con la norma $\|\;\|_1$, allora lo stesso deve
essere valido per la norma $\|\;\|_2$. Per definizione vogliamo che
$\forall \delta>0$ $\exists N$ tale che $\|v-v_i\|_2<\delta$ per $i>N$.
Sappiamo che $\forall \delta/K$ $\exists n$ tale che $\|v-v_i\|_1<\delta/K$.
Ma $\|v-v_i\|_2\leq K\|v-v_i\|_1<\delta$. Per dimostrare l'inverso usiamo
$\|v\|_1<\frac{1}{k}\|v\|_2$.
Con ragionamenti analoghi si vede che tutte le altre nozioni topologiche
(limite, continuità,\ldots) sono equivalenti per norme equivalenti.
\end{proof}
Più in generale abbiamo che in dimensione finita tutte le norme sono
equivalenti, per cui per trovare controesempi di norme non equivalenti
dobbiamo guardare agli spazi infiniti.
\begin{example}[norme non equivalenti]
	
Le norme $\|f\|_1=max|f(x)|$ e $\|f\|_2=\int_{a}^b|f(x)dx$ non possono essere
equivalenti. Ad esempio:
\[
f_n(x)=\begin{cases}
0 \qquad &0 \leq x \leq c - 1/n\\
n(x-c+1/n) \qquad &c - 1/n \leq x \leq c\\
n(-x+c+1/n)\qquad &c \leq x \leq c + 1/n\\
0\qquad &c + 1/n \leq x \leq b
\end{cases}
\]
abbiamo $\|f_n\|_1=1$ $\forall n$ mentre $\|f_n\|_2=1/n\to 0$ per
$n\to \infty$ quindi non può esistere $k$ tale che
$\|f_n\|_1\leq k \|f_n\|_2$ $\forall n$.
\end{example}

Vediamo altre nozioni di topologia.\\
\begin{definition}[Palla aperta]
Dato un elemento $v \in V$, definiamo $B(v,R) \subset V$, la \emph{palla aperta}
di raggio $R$ attorno a $v$ come l'insieme degli elementi $w \in V$ tali
che $\|v-w\| < R$. 
\end{definition}

\begin{definition}[Palla chiusa]
$\overline{B(v,R)}$ è la \emph{palla chiusa} data dagli elementi
per cui $\|v-w\| \leq R$.
\end{definition}

\begin{definition}[aperto]
Un insieme $A \subset V$ è \emph{aperto} se per ogni suo elemento $v\in A$
esiste un $\delta$ tale che l'intera palla aperta $B(v, \delta)$ è contenuta
in $A$.
\end{definition}

\begin{definition}[chiuso]
Un insieme $C \subset V$ è \emph{chiuso} se il suo complementare è aperto.
\end{definition}

\begin{theorem}
La palla aperta è un insieme aperto.
\end{theorem}
\begin{proof}
Dato $w$ con $\|w-v\|=r<R$, prendiamo
$\delta < R-r$, se $t \in B(w, \delta)$, $\|t-w\| < \delta$. Quindi $\|t-v\|
\leq \|z-w\| + \|w-v\| = \delta +r < R$ quindi $t \in B(v,R)$. 
\end{proof}
In maniera analoga si dimostra che $\overline{B(v,R)}$ è chiuso.
\begin{theorem}
Un insieme è chiuso $\iff$ contiene tutti i suoi punti di accumulazione:
$C = \overline{C}$.
\end{theorem}
\begin{proof}
Se $v \notin C$ allora $v \in C^c$ cioè $v$ appartiene al
complementare. Essendo $C^c$ aperto esiste una palla attorno a $v$ in cui
nessun elemento appartiene a $C$, quindi $v$ non può essere un punto di
accumulazione di $C$.
\end{proof}
