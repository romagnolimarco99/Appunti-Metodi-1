\chapter{Prodotto scalare e Spazi di Hilbert}

\begin{definition}{Prodotto Scalare}
\label{prod_scal}
Prendiamo uno spazio vettoriale $V$. Un prodotto scalare 
\begin{equation*}
(\,,\,):V \times V \longmapsto \mathbb{C}
\end{equation*}  

è definito dalle seguenti proprietà:
\begin{enumerate}
\item $(v,v) \geq 0$ e $(v,v) = 0 \leftrightarrow v = 0$
\item $(v,w)*=(w,v)$
\item $(v,\lambda w) = \lambda(v,w)$
\item $(v,w_1+w_2) = (v,w_1) + (v,w_2)$
\end{enumerate} 

\end{definition}

Possiamo definire la seguente funzione $\norm{v}  = \sqrt{(v,v)}$ e dimostrare
che è una norma.

\begin{theorem}\label{th: dis_sch}
Sia $V$ uno spazio vettoriale con prodotto scalare $(,)$.
Sia $\norm{v} = \sqrt{(v,v)}$. Allora $\norm{v}$ è una norma per cui,
per ogni $v,w \in V$, vale la disuguaglianza di Cauchy-Schwarz 
\addcontentsline{toc}{subsection}{disuguaglianza di Cauchy}
\begin{equation}\label{eq: dis_sch}
\abs{(v,w)} \leq \norm{v} \cdot \norm{w}.
\end{equation} 
\end{theorem}

\begin{proof}
Si vede immediatamente che $\norm{v}$ è una norma ben definita, grazie al
fatto che il prodotto scalare è definito positivo per costruzione. \\
Per cui $\norm{v - \lambda(v,w)w} \geq 0$ con $\lambda \in \RR$. 
Quindi
\begin{gather*}
(v - \lambda(v,w)w,v - \lambda(v,w)w) = \\
=\norm{v}^2 - 2\lambda\abs{(v,w)}^2 +\lambda^2 \abs{(v,w)}^2 \norm{w}^2 \geq 0 ;
\end{gather*}

Per cui
\begin{align*}
4\abs{(v,w)}\left(\abs{(v,w)}^2 - \norm{v}^2 \norm{w}^2\right) \leq 0
\end{align*}

da cui la disuguaglianza.
\end{proof}
\begin{remark}
La disuguaglianza in \ref{th: dis_sch} ci permette di verificare che questa
norma soddisfa la disuguaglianza triangolare (proprietà $3.$ della norma):
\begin{align*}
\|v + w\|^2 &= (v + w, v + w) = \|v\|^2 + (v, w) + (w, v) \|w\|^2 
\leq \|v\|^2 + 2 \abs{(w, v)} + \|w\|^2 \\
&\leq \|v\|^2 + 2 \|v\| \cdot \|w\| + \|w\|^2 = \left( \|v\| + \|w\| \right)^2
\end{align*}
\end{remark}
Quindi uno spazio dotato di prodotto scalare è anche uno spazio normato.
Se lo spazio è anche completo si chiama spazio di Hilbert
$\left(\mathcal{H}\right)$. Un esempio di spazio di Hilbert è $\mathbb{C}^n$
con il prodotto scalare $(z,w) = \sqrt{\sum_i z_i^\ast w_i}$

\begin{theorem}[Regola del parallelogrammo]\label{th: parallelogram}
Una norma indotta da un prodotto scalare soddisfa l'uguaglianza
\[
\|v - w\|^2 + \|v + w\|^2 = 2\|v\|^2 + 2\|w\|^2
.\]
detta regola del parallelogrammo.
\end{theorem}
\begin{proof}
facendo il conto esplicito si annullano i termini misti del prodotto $(\, ,\,)$
\begin{align*}
	(v - w, v - w) + (v + w, v + w) = \|v\|^2 - (v, w) - (w, v) + \|w\|^2
	+ \|v\|^2 + (v, w) + (w, v) + \|w\|^2	
\end{align*}
\end{proof}

\begin{remark}
Se la regola sopra è soddisfatta, vale anche l'implicazione inversa del teorema
\ref{th: parallelogram}: si può sempre trovare il prodotto scalare da cui la
norma discende (Teorema di Von Neumann).
\end{remark}
