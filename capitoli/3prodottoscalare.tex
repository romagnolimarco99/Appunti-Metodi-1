\chapter{Prodotto scalare e Spazi di Hilbert}

\begin{definition}{Prodotto Scalare}
\label{prod_scal}
Prendiamo uno spazio vettoriale $V$. Un prodotto scalare 
\begin{equation*}
(\,,\,):V \times V \longmapsto \mathbb{C}
\end{equation*}  

è definito dalle seguenti proprietà:
\begin{enumerate}
\item $(v,v) \geq 0$ e $(v,v) = 0 \leftrightarrow v = 0$
\item $(v,w)*=(w,v)$
\item $(v,\lambda w) = \lambda(v,w)$
\item $(v,w_1+w_2) = (v,w_1) + (v,w_2)$
\end{enumerate} 

\end{definition}

Possiamo definire la seguente funzione $\norm{v}  = \sqrt{(v,v)}$ e dimostrare che è una norma.

\begin{theorem}
Sia $V$ uno spazio vettoriale con prodotto scalare $(,)$. Sia $\norm{v} = \sqrt{(v,v)}$. Allora $\norm{v}$ è una norma per cui, per ogni $v,w \in V$, vale la disuguaglianza di Cauchy-Schwarz \index{disuguaglianza di Cauchy}
\begin{equation}
\label{dis_sch}
\abs{(v,w)} \leq \norm{v} \cdot \norm{w}.
\end{equation} 
\end{theorem}

\begin{proof}
Si vede immediatamente che $\norm{v}$ è una norma ben definita, grazie al fatto che il prodotto scalare è definito positivo per costruzione. \\
Per cui $\norm{v - \lambda(v,w)w} \geq 0$ con $\lambda \in \mathbb{R}$. 
Quindi
\begin{gather*}
(v - \lambda(v,w)w,v - \lambda(v,w)w) = \\
=\norm{v}^2 - 2\lambda\abs{(v,w)}^2 +\lambda^2 \abs{(v,w)}^2 \norm{w}^2 \geq 0 .
\end{gather*}

Per cui
\begin{align*}
4\abs{(v,w)}\left(\abs{(v,w)}^2 - \norm{v}^2 \norm{w}^2\right) \leq 0
\end{align*}

da cui la disuguaglianza.\\


\end{proof}

Quindi uno spazio dotato di prodotto scalare è anche uno spazio normato. Se lo spazio è anche completo si chiama spazio di Hilbert $\left(\mathcal{H}\right)$. Un esempio di spazio di Hilbert è $\mathbb{C}^n$ con il prodotto scalare $(z,w) = \sqrt{\sum_i z_i^\ast w_i}$