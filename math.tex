\declaretheoremstyle[
spaceabove=6pt, spacebelow=6pt,
headfont=\normalfont\bfseries\itshape,
notefont=\mdseries, notebraces={(}{)},
bodyfont=\normalfont,
postheadspace=1em,
qed=,
]{exercise_style}

\declaretheoremstyle[
spaceabove=6pt, spacebelow=6pt,
postheadspace=1em,
qed=,
]{theorem_style}

\declaretheoremstyle[
spaceabove=6pt, spacebelow=6pt,
postheadspace=1em,
qed=,
]{axiom_style}

\declaretheorem[name=Teorema,numberwithin=chapter]{theorem}
\declaretheorem[name=Lemma,sibling=theorem]{lemma}
\declaretheorem[name=Proposizione,sibling=theorem]{proposition}
\declaretheorem[name=Corollario,sibling=theorem]{corollary}
\declaretheorem[name=Paradosso,sibling=theorem]{paradox}
\declaretheorem[style=axiom_style,name=Assioma,sibling=theorem]{axiom}
\declaretheorem[name=Definizione,sibling=theorem]{definition}
\declaretheorem[style=exercise_style,name=Esempio,sibling=theorem]{example}
\declaretheorem[style=exercise_style,name=Esercizio,sibling=theorem]{exercise}
\declaretheorem[style=exercise_style,name=Osservazione,sibling=theorem]{remark}

\iffalse
\newtheorem{theorem}{Teorema}[section]
\newtheorem{corollary}{Corollario}[theorem]
\newtheorem{lemma}[theorem]{Lemma}
\newtheorem{definition}{Definizione}[section]
\fi
%\renewcommand\qedsymbol{$\smiley$}

\newcommand{\abs}[1]{{\left|#1\right|}}
\newcommand{\norm}[1]{{\|#1\|}}
\DeclareMathOperator{\Imaginarypart}{Im}
\renewcommand{\Im}{\Imaginarypart}
\DeclareMathOperator{\Realpart}{Re}
\renewcommand{\Re}{\Realpart}

% greek letters
\newcommand{\eps}{\varepsilon}
\renewcommand{\phi}{\varphi}

% blackboard letters
\newcommand{\CC}{\mathbb C}
\newcommand{\HH}{\mathbb H}
\newcommand{\KK}{\mathbb K}
\newcommand{\NN}{\mathbb N}
\newcommand{\QQ}{\mathbb Q}
\newcommand{\RR}{\mathbb R}
\newcommand{\TT}{\mathbb T}
\newcommand{\ZZ}{\mathbb Z}

% Upright d in math mode (for differentials).
% -> d
\newcommand{\ud}{\mathrm{d}}

% Differential.
% -> dx
\newcommand{\diff}[1][x]{\,\ud{#1}}

% Base command for defining derivatives.
% -> df/dx or d^kf/dx^k
\newcommand{\basederivative}[4][]{%
  \displaystyle%
  \ifx\\#1\\\frac{#4#2}{#4#3}%
  \else%
  \frac{#4^#1#2}{#4#3^#1}%
  \fi%
}

% Total derivative.
% -> df/dx(x) or d^kf/dx^k(x)
\newcommand{\td}[4][]{%
  \basederivative[#1]{#2}{#3}{\ud}%
  \ifx\\#4\\%
  \else%
  \mkern-4mu\left(#4\right)%
  \fi%
}

% Partial derivative.
% -> df/dx(x) or d^kf/dx^k(x)
\newcommand{\pd}[4][]{%
  \basederivative[#1]{#2}{#3}{\partial}%
  \ifx\\#4\\%
  \else%
  \mkern-4mu\left(#4\right)%
  \fi%
}
